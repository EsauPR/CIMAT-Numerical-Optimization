\documentclass[conference]{IEEEtran}
\IEEEoverridecommandlockouts
% The preceding line is only needed to identify funding in the first footnote. If that is unneeded, please comment it out.
\usepackage{cite}
\usepackage{amsmath,amssymb,amsfonts}
\usepackage{algorithmic}
\usepackage{graphicx}
\usepackage{textcomp}
\usepackage{xcolor}
\def\BibTeX{{\rm B\kern-.05em{\sc i\kern-.025em b}\kern-.08em
    T\kern-.1667em\lower.7ex\hbox{E}\kern-.125emX}}

\ifCLASSINFOpdf
\else
    \usepackage[dvips]{graphicx}
\fi
\usepackage{url}
\usepackage{listings}
% \usepackage[margin=1in]{geometry}
\usepackage{amsmath,amsthm,amssymb}
\usepackage{amsmath,amsthm,amssymb}
\usepackage[spanish]{babel} %Castellanización
\usepackage[T1]{fontenc} %escribe lo del teclado
\usepackage[utf8]{inputenc} %Reconoce algunos símbolos
\usepackage{lmodern} %optimiza algunas fuentes
\usepackage{blkarray}
\graphicspath{ {images/} }
\usepackage{hyperref} % Uso de links

\newcommand{\N}{\mathbb{N}}
\newcommand{\Z}{\mathbb{Z}}
\usepackage{float}
\newenvironment{theorem}[2][Theorem]{\begin{trivlist}
\item[\hskip \labelsep {\bfseries #1}\hskip \labelsep {\bfseries #2.}]}{\end{trivlist}}
\newenvironment{lemma}[2][Lemma]{\begin{trivlist}
\item[\hskip \labelsep {\bfseries #1}\hskip \labelsep {\bfseries #2.}]}{\end{trivlist}}
\newenvironment{exercise}[2][Exercise]{\begin{trivlist}
\item[\hskip \labelsep {\bfseries #1}\hskip \labelsep {\bfseries #2.}]}{\end{trivlist}}
\newenvironment{problem}[2][Problem]{\begin{trivlist}
\item[\hskip \labelsep {\bfseries #1}\hskip \labelsep {\bfseries #2.}]}{\end{trivlist}}
\newenvironment{question}[2][Question]{\begin{trivlist}
\item[\hskip \labelsep {\bfseries #1}\hskip \labelsep {\bfseries #2.}]}{\end{trivlist}}
\newenvironment{corollary}[2][Corollary]{\begin{trivlist}
\item[\hskip \labelsep {\bfseries #1}\hskip \labelsep {\bfseries #2.}]}{\end{trivlist}}
\newcommand*{\defeq}{\stackrel{\text{def}}{=}}
\newenvironment{solution}{\begin{proof}[Solution]}{\end{proof}}
\hyphenation{op-tical net-works semi-conduc-tor}

\newcommand{\argmax}{\operatornamewithlimits{argmax}}


\usepackage[ruled,vlined]{algorithm2e}


\begin{document}

\title{Tarea 8. Optimización, Introducción al Cálculo Variacional}

\author{\IEEEauthorblockN{Oscar Esaú Peralta Rosales}
\IEEEauthorblockA{\textit{Maestría en Computación} \\
\textit{Centro de Investigación en Matemáticas}}
}

\maketitle


\subsection*{1. Usa la ecuación de Euler-Lagrange para buscar los extremos de las siguientes funcionales}

\begin{equation}
    J[y] = \int_{a}^{b} \big( xy\prime + (y\prime)^2 \big)dx
\end{equation}

\begin{equation}
    J[y] = \int_{a}^{b} (1+x) (y\prime)^2 dx
\end{equation}

\textbf{Solución}

La ecuación de Euler-Lagrange está dado por:

\begin{equation*}
    \frac{\partial f}{\partial y} - \frac{d}{dx} \Big(\frac{\partial f}{\partial y\prime}\Big) = 0
\end{equation*}

Para la función (1) tenemos que: $\frac{\partial f}{\partial y} = 0$,
$\frac{\partial f}{\partial y\prime} = x + 2y\prime$ y
$\frac{d}{dx} \Big(\frac{\partial f}{\partial y\prime}\Big) = 1 + 2y\prime\prime$.

Usando la ecuación de Euler-Lagrange tenemos: $1 + 2y\prime\prime = 0$ y por tanto
$y\prime\prime = - \frac{1}{2}$.

ntegrando ambos miembros:

\begin{equation*}
    \int \frac{d y\prime}{dx} dx = - \frac{1}{2} \int dx
\end{equation*}

\begin{equation*}
    \int d y\prime = - \frac{1}{2} \int dx
\end{equation*}

\begin{equation*}
    y\prime = - \frac{1}{2} x + c_1
\end{equation*}

Finalmente, integrando nuevamente:

\begin{equation*}
    \int \frac{dy}{dx} dx = \int (- \frac{1}{2} x + c_1) dx
\end{equation*}

\begin{equation*}
    \int dy = \int (- \frac{1}{2} x + c_1) dx
\end{equation*}

\begin{equation*}
    y = - \frac{1}{4} x^2 + c_1x + c_2
\end{equation*}


Para la función (2) tenemos que: $\frac{\partial f}{\partial y} = 0$,
$\frac{\partial f}{\partial y\prime} = 2(1 + x)y\prime$ y
$\frac{d}{dx} \Big(\frac{\partial f}{\partial y\prime}\Big) = 2y\prime + 2(1+x)y\prime\prime$.

Usando la ecuación de Euler-Lagrange tenemos: $2y\prime + 2(1+x)y\prime\prime = 0$ y por tanto
$\frac{y\prime\prime}{y\prime} = - \frac{1}{1+x}$.


Integrando ambos miembros:

\begin{equation*}
    \int \frac{1}{y\prime} \frac{dy\prime}{dx} dx = - \int \frac{1}{1+x} dx
\end{equation*}

\begin{equation*}
    \int \frac{1}{y\prime} dy\prime = - \int \frac{1}{1+x} dx
\end{equation*}

\begin{equation*}
    ln(y\prime) = ln(\frac{1}{1+x}) + c_1
\end{equation*}

Aplicando la exponencial: $y\prime = \frac{1}{1+x} e^{c_1}$, hacemos $e^{c_1} = c2$ e integrando
nuevamente:

\begin{equation*}
    \int \frac{dy}{dx} dx = c_2 \int \frac{1}{1+x} dx
\end{equation*}

\begin{equation*}
    y = c_2 ln(1+x) + c_3
\end{equation*}




\subsection*{2. Derivar las ecuaciones de Euler-Lagrange usando el método de Lagrange de}

\begin{equation}
    \int_x \int_y F(x, y, f, f_x, f_y) dx dy
\end{equation}

\begin{equation}
    \int_x \int_y F(x, y, u, v, u_x, v_x, u_y, v_y) dx dy
\end{equation}

donde $f,u,v: \mathcal{R}^2 \rightarrow \mathcal{R}$ y $f_x = \frac{\partial f}{\partial x}$,
$f_y = \frac{\partial f}{\partial y}$, $u_x = \frac{\partial u}{\partial x}$,
$u_y = \frac{\partial u}{\partial y}$, $v_x = \frac{\partial v}{\partial x}$ y
$v_y = \frac{\partial v}{\partial y}$.\\

\textbf{Solución}

Para la función (3) introducimos la variación $W(x,y) = f(x,y) + \epsilon \eta(x,y)$,
con $\eta(x,y) = 0$ sobre todo $\mathcal{C}$ el cual es la curva límite del dominio de integración.

\begin{equation*}
    I(\epsilon) = \int_x \int_y F(x,y,W,W_x,W_y)dx dy
\end{equation*}

Diferenciamos con respecto a $\epsilon$

\begin{equation*}
    I\prime(\epsilon) = \int_x \int_y \Big(
    \frac{\partial F}{\partial W} \frac{\partial W}{\partial \epsilon}
    + \frac{\partial F}{\partial W_x} \frac{\partial W_x}{\partial \epsilon}
    + \frac{\partial F}{\partial W_y} \frac{\partial W_y}{\partial \epsilon}
    \Big) dx dy
\end{equation*}

usando $\frac{\partial W}{\partial \epsilon} = \eta$,
$\frac{\partial W_x}{\partial \epsilon} = \eta_x$ y
$\frac{\partial W_y}{\partial \epsilon} = \eta_y$

\begin{equation*}
    I\prime(\epsilon) = \int_x \int_y \Big(
    \frac{\partial F}{\partial W} \eta
    + \frac{\partial F}{\partial W_x} \eta_x
    + \frac{\partial F}{\partial W_y} \eta_y
    \Big) dx dy
\end{equation*}

evaluando en $\epsilon=0$

\begin{equation*}
    I\prime(0) = \int_x \int_y \Big(
    \frac{\partial F}{\partial f} \eta
    + \frac{\partial F}{\partial f_x} \eta_x
    + \frac{\partial F}{\partial f_y} \eta_y
    \Big) dx dy = 0
\end{equation*}

puesto que $f(x,y)$ es un extremo para $\epsilon = 0$. Usando el teoremoa de Green para reexpresar

los dos ultimos sumandos tenemos

$$
    \int_x \int_y \eta \Big[
    \frac{\partial F}{\partial f}
    - \frac{\partial }{\partial x} \Big(\frac{\partial F}{\partial f_x}\Big)
    - \frac{\partial }{\partial y} \Big(\frac{\partial F}{\partial f_y}\Big)
    \Big] dx dy +
$$
$$
    \int_C \eta \Big(
    \frac{\partial F}{\partial f_x} \frac{y}{ds}
    - \Big(\frac{\partial F}{\partial f_y} \frac{x}{ds}
    \Big) ds = 0
$$

Como $\eta(x,y) = 0$ en todo $\mathcal{C}$ entonces

$$
    \int_x \int_y \eta \Big[
    \frac{\partial F}{\partial f}
    - \frac{\partial }{\partial x} \Big(\frac{\partial F}{\partial f_x}\Big)
    - \frac{\partial }{\partial y} \Big(\frac{\partial F}{\partial f_y}\Big)
    \Big] dx dy = 0
$$

Así la función extrema $f(x,y)$ debe cumplir que

\begin{equation}
    \frac{\partial F}{\partial f}
    - \frac{\partial }{\partial x} \Big(\frac{\partial F}{\partial f_x}\Big)
    - \frac{\partial }{\partial y} \Big(\frac{\partial F}{\partial f_y}\Big)
        = 0
\end{equation}

Para la función 4, proponemos dos variaciones con las mismas condiciones anteriores:
$W(x,y) = u(x,y) + \epsilon_1 \eta(x,y)$ y $Z(x,y) = v(x,y) + \epsilon_2 \phi(x,y)$, así tenemos
que:

$$
I(\epsilon_1, \epsilon_2) = \int_x \int_y F(x, y, W, Z, W_x, Z_x, W_y, Z_y) dx dy
$$

Derivamos con respecto a $\epsilon_1$ y $epsilon_2$

$$
    I\prime(\epsilon) =
    \begin{bmatrix}
        \int_x \int_y \Big(
            \frac{\partial F}{\partial W} \eta
            + \frac{\partial F}{\partial W_x} \eta_x
            + \frac{\partial F}{\partial W_y} \eta_y
            \Big) dx dy \\
        \int_x \int_y \Big(
            \frac{\partial F}{\partial Z} \eta
            + \frac{\partial F}{\partial Z_x} \eta_x
            + \frac{\partial F}{\partial Z_y} \eta_y
            \Big) dx dy
    \end{bmatrix}
$$

Evaluando en $I\prime(0,0)$

$$
    \begin{bmatrix}
        \int_x \int_y \Big(
            \frac{\partial F}{\partial W} \eta
            + \frac{\partial F}{\partial u_x} \eta_x
            + \frac{\partial F}{\partial u_y} \eta_y
            \Big) dx dy \\
        \int_x \int_y \Big(
            \frac{\partial F}{\partial Z} \eta
            + \frac{\partial F}{\partial v_x} \eta_x
            + \frac{\partial F}{\partial v_y} \eta_y
            \Big) dx dy
        \end{bmatrix} = \begin{bmatrix}
            0\\
            0
        \end{bmatrix}
$$

Usando el teorema de Green llegamos a que:

$$
    \begin{bmatrix}
        \int_x \int_y \eta \Big(
        \frac{\partial F}{\partial v}
        - \frac{\partial }{\partial x} \big(\frac{\partial F}{\partial u_x}\big)
        - \frac{\partial }{\partial y} \big(\frac{\partial F}{\partial u_y}\big)
        \Big) dx dy\\
        \int_x \int_y \eta \Big(
        \frac{\partial F}{\partial v}
        - \frac{\partial }{\partial x} \big(\frac{\partial F}{\partial v_x}\big)
        - \frac{\partial }{\partial y} \big(\frac{\partial F}{\partial v_y}\big)
        \Big) dx dy
        \end{bmatrix} = \begin{bmatrix}
            0\\
            0
        \end{bmatrix}
$$

Así la funciones extremas $u(x, y)$ y $v(x,y)$ debe cumplir que:


\begin{equation}
    \begin{bmatrix}
    \frac{\partial F}{\partial v}
    - \frac{\partial }{\partial x} \big(\frac{\partial F}{\partial u_x}\big)
    - \frac{\partial }{\partial y} \big(\frac{\partial F}{\partial u_y}\big)\\
    \frac{\partial F}{\partial v}
    - \frac{\partial }{\partial x} \big(\frac{\partial F}{\partial v_x}\big)
    - \frac{\partial }{\partial y} \big(\frac{\partial F}{\partial v_y}\big)\\
    \end{bmatrix} = \begin{bmatrix}
        0\\
        0
    \end{bmatrix}
\end{equation}


\subsection*{3. Obtener la ecuaciones de Euler-Lagrange de:}

\begin{equation}
    \int_x \int_y ((f-g)^2 + \lambda ||\nabla f||^2) dx dy
\end{equation}

\begin{equation}
    \int_x \int_y ( (p-q-p_xu-q_xv)^2 + \lambda (||\nabla u||^2 + ||\nabla v||^2) ) dx dy
\end{equation}

donde $f,u,v: \mathcal{R}^2 \rightarrow \mathcal{R}$ y
$g,p,q: \mathcal{R}^2 \rightarrow \mathcal{R}$ son funciones dadas.


\textbf{Solución}

Notemos que podemos reexpresar la ecuación (7) a través del producto interior como

$$
\int_x \int_y ((f-g)^2 + \lambda (f_x^2 - f_y^2)) dx dy
$$

Usando (5) tenemos que $\frac{\partial F}{\partial f} = 2(f-g)$,
$\frac{\partial F}{\partial f_x} = 2\lambda f_x$,
$\frac{\partial F}{\partial f_y} = 2\lambda f_y$,
$\frac{\partial}{\partial x}\frac{\partial F}{\partial f_x} = 2\lambda f_{xx}$
$\frac{\partial}{\partial y}\frac{\partial F}{\partial f_y} = 2\lambda f_{yy}$, así, obtenemos que
debe cumplir con:

$$
(f-g) - \lambda(f_{xx} + f_{yy}) = 0
$$

Para la ecuación (8) tambien podemos reexpresarla usando producto interior como:

$$
    \int_x \int_y ( (p-q-p_xu-q_xv)^2 + \lambda (u_x^2 + u_y^2 + v_x^2 + v_y^2) ) dx dy
$$

Usando (7) tenemos que $\frac{\partial F}{\partial v} = -2q_x(p-q-p_xu-q_xv)$,
$\frac{\partial}{\partial x}\frac{\partial F}{\partial v_x} = 2\lambda v_{xx}$
$\frac{\partial}{\partial y}\frac{\partial F}{\partial v_y} = 2\lambda v_{yy}$, así, obtenemos que
debe cumplir con:

$$
q_x(p-q-p_xu-q_xv) + \lambda (v_{xx} + v_{yy}) = 0
$$

por otro lado $\frac{\partial F}{\partial u} = -2p_x(p-q-p_xu-q_xv)$,
$\frac{\partial}{\partial x}\frac{\partial F}{\partial u_x} = 2\lambda u_{xx}$
$\frac{\partial}{\partial y}\frac{\partial F}{\partial u_y} = 2\lambda u_{yy}$, así, obtenemos
que también debe cumplir con:

$$
p_x(p-q-p_xu-q_xv) + \lambda (u_{xx} + u_{yy}) = 0
$$

\end{document}
